% Created 2018-07-13 Fri 11:20
% Intended LaTeX compiler: pdflatex
\documentclass[11pt]{article}
\usepackage[utf8]{inputenc}
\usepackage[T1]{fontenc}
\usepackage{graphicx}
\usepackage{grffile}
\usepackage{longtable}
\usepackage{wrapfig}
\usepackage{rotating}
\usepackage[normalem]{ulem}
\usepackage{amsmath}
\usepackage{textcomp}
\usepackage{amssymb}
\usepackage{capt-of}
\usepackage{hyperref}
\author{Anirudh C}
\date{}
\title{1D Advection Equation}
\hypersetup{
 pdfauthor={Anirudh C},
 pdftitle={1D Advection Equation},
 pdfkeywords={},
 pdfsubject={},
 pdfcreator={Emacs 26.1 (Org mode 9.1.9)}, 
 pdflang={English}}
\begin{document}

\maketitle
\tableofcontents

\section{Description}
\label{sec:org7c9dfc4}
Consider the 1D advection equation,
$$u_t + a u_x = 0$$
defined on the domain \([a,b]\)

With periodic boundary conditions, that is,
$$u(x+1,t) = u(x,t)$$
\section{Discretization}
\label{sec:org42636e3}
Partition the domain into \(N\) points, such that
$$x_i = i \Delta x$$
where, \(\Delta x\) is the partition size defined as,
$$\Delta x = \frac{b - a}{N-1}$$
The time is partitioned so that,
$$t_n = n \Delta t$$
Where \(\Delta t\) is the time interval

Let \(U_i^n\) be the approximation to the function \(u\), that is,
$$U_i^n \approx u(x_i,t_n)$$

By the finite difference method the time derivative is approximated using a forward difference scheme.
$$\frac{\partial}{\partial t} u(x_i,t) \bigg|_{t=t_n} \Rightarrow \frac{U_i^{n+1} - U_i^n}{\Delta t}$$
The space derivative can be approximated in three different schemes:
\subsection{Forward Difference}
\label{sec:orgaeab8ce}
\subsubsection{Description}
\label{sec:orgb9f309e}
$$\frac{\partial}{\partial x} u(x,t_n) \bigg|_{x=x_i} \Rightarrow \frac{U_{i+1}^{n+1} - U_i^n}{\Delta x}$$
Thus we get,
$$\frac{U_i^{n+1} - U_i^n}{\Delta t} + a \frac{U_{i+1}^{n+1} - U_i^n}{\Delta x} = 0$$
Define,
$$\sigma = \frac{a \Delta t}{\Delta x}$$
Rearranging we get,
$$U_i^{n+1} = \left( 1 + \sigma \right) U_i^n - \sigma U_{i+1}^n$$
where,
$$i=0,1,\ldots,N-2 \qquad n = 0,1,2,\ldots$$
With the periodic boundary condition we define,
$$U_{N-1}^{n+1} = \left( 1 + \sigma \right) U_{N-1}^n - \sigma U_{1}^n$$
\subsubsection{Stability}
\label{sec:orge836e3c}

Using Fourier Analysis we get, the solution
$$U_i^n = \beta^n e^{i l x_i}$$
Substituting in the scheme we get,
$$\beta^{n+1} e^{i l x_i} = \left( 1 + \sigma \right) \beta^n e^{i l x_i} - \sigma \beta^n e^{i l x_{i+1}}$$
Thus,
$$\beta = 1 + \sigma - \sigma e^{i l \Delta x}$$
$$\beta = 1 + \sigma - \sigma \left( cos(l \Delta x) + i sin(\Delta x) \right)$$
$$\lvert \beta \rvert = 1 + 2 \sigma \left( \sigma +1 \right) \left( 1 - cos(l \Delta x) \right)$$

\begin{itemize}
\item If \(a<0\)

$$\sigma < 0$$
For a stable solution we need \(\lvert \beta \rvert < 1\)
$$1 + 2 \sigma \left( \sigma +1 \right) \left( 1 - cos(l \Delta x) \right) <1$$
$$2 \sigma \left( \sigma +1 \right) \left( 1 - cos(l \Delta x) \right) <0$$
Since \(\sigma <0\) and \(1 - cos(l \Delta x) > 0\)
$$\sigma + 1 > 0 \Rightarrow \sigma > -1$$
Thus if \(a<0\) the forward scheme is stable iff \(\sigma > -1\)

\item If \(a>0\)

$$\sigma > 0$$
For a stable solution we need \(\lvert \beta \rvert < 1\)
$$1 + 2 \sigma \left( \sigma +1 \right) \left( 1 - cos(l \Delta x) \right) <1$$
$$2 \sigma \left( \sigma +1 \right) \left( 1 - cos(l \Delta x) \right) <0$$
Since \(\sigma >0\) and \(1 - cos(l \Delta x) > 0\)
$$\sigma + 1<0 \Rightarrow \sigma < -1$$
This is not possible

Thus if \(a>0\) the forward scheme is unconditionally unstable.
\end{itemize}
\subsection{Backward Difference}
\label{sec:org7518942}
\subsubsection{Description}
\label{sec:org8271bcc}
$$\frac{\partial}{\partial x} u(x,t_n) \bigg|_{x=x_i} \Rightarrow \frac{U_i^{n+1} - U_{i-1}^n}{\Delta x}$$
Thus we get,
$$\frac{U_i^{n+1} - U_i^n}{\Delta t} + a \frac{U_i^{n+1} - U_{i-1}^n}{\Delta x} = 0$$
Define,
$$\sigma = \frac{a \Delta t}{\Delta x}$$
Rearranging we get,
$$U_i^{n+1} = \left( 1 - \sigma \right) U_i^n + \sigma U_{i-1}^n$$
where,
$$i=1,2,\ldots,N-1 \qquad n = 0,1,2,\ldots$$
With the periodic boundary condition we define,
$$U_{0}^{n+1} = \left( 1 - \sigma \right) U_{0}^n + \sigma U_{N-2}^n$$
\subsubsection{Stability}
\label{sec:org09fdb0e}

Using Fourier Analysis we get, the solution
$$U_i^n = \beta^n e^{i l x_i}$$
Substituting in the scheme we get,
$$\beta^{n+1} e^{i l x_i} = \left( 1 - \sigma \right) \beta^n e^{i l x_i} + \sigma \beta^n e^{i l x_{i-1}}$$
Thus,
$$\beta = 1 - \sigma + \sigma e^{- i l \Delta x}$$
$$\beta = 1 - \sigma + \sigma cos(l \Delta x) - i \sigma sin(l \Delta x)$$
$$\lvert \beta \rvert = 1 + 2 \sigma \left( \sigma -1 \right) \left( 1 - cos(l \Delta x) \right)$$

\begin{itemize}
\item If \(a<0\)

$$\sigma < 0$$
For a stable solution we need \(\lvert \beta \rvert < 1\)
$$1 + 2 \sigma \left( \sigma -1 \right) \left( 1 - cos(l \Delta x) \right) <1$$
$$2 \sigma \left( \sigma -1 \right) \left( 1 - cos(l \Delta x) \right) <0$$
Since \(\sigma <0\) and \(1 - cos(l \Delta x) > 0\)
$$\sigma - 1 > 0 \Rightarrow \sigma > 1$$
This is not possible

Thus if \(a<0\) the backward scheme is unconditionally unstable.

\item If \(a>0\)

$$\sigma > 0$$
For a stable solution we need \(\lvert \beta \rvert < 1\)
$$1 + 2 \sigma \left( \sigma -1 \right) \left( 1 - cos(l \Delta x) \right) <1$$
$$2 \sigma \left( \sigma -1 \right) \left( 1 - cos(l \Delta x) \right) <0$$
Since \(\sigma >0\) and \(1 - cos(l \Delta x) > 0\)
$$\sigma -1 < 0 \Rightarrow \sigma < 1$$

Thus if \(a>0\) the backward scheme is stable iff \(\sigma < 1\).
\end{itemize}
\subsection{Central Difference}
\label{sec:orgb1047b1}
\subsubsection{Description}
\label{sec:orge42c561}
$$\frac{\partial}{\partial x} u(x,t_n) \bigg|_{x=x_i} \Rightarrow \frac{U_{i+1}^{n+1} - U_{i-1}^n}{\Delta x}$$
Thus we get,
$$\frac{U_i^{n+1} - U_i^n}{\Delta t} + a \frac{U_{i+1}^{n+1} - U_{i-1}^n}{\Delta x} = 0$$
Define,
$$\sigma = \frac{a \Delta t}{\Delta x}$$
Rearranging we get,
$$U_i^{n+1} = U_i^n - \frac{\sigma}{2} U_{i+1}^n + \frac{\sigma}{2} U_{i-1}^n$$
where,
$$i=1,2,\ldots,N-2 \qquad n = 0,1,2,\ldots$$
With the periodic boundary condition we define,
$$U_0^{n+1} = U_0^n - \frac{\sigma}{2} U_{1}^n + \frac{\sigma}{2} U_{N-2}^n$$
$$U_{N-1}^{n+1} = U_{N-1}^n - \frac{\sigma}{2} U_{1}^n + \frac{\sigma}{2} U_{N-2}^n$$
\subsubsection{Stability}
\label{sec:orga1a91a9}

Using Fourier Analysis we get, the solution
$$U_i^n = \beta^n e^{i l x_i}$$
Substituting in the scheme we get,
$$\beta^{n+1} e^{i l x_i} = \beta^n e^{i l x_i} - \frac{\sigma}{2} \beta^n e^{i l x_{i+1}} + \frac{\sigma}{2} \beta^{n} e^{i l x_{i-1}}$$
Thus,
$$\beta = 1 - \sigma/2 e^{i l \Delta x} + \sigma/2 e^{- i l \Delta x}$$
$$\lvert \beta \rvert = 1 + \sigma^{2} \sin^{2} (l \Delta x)$$
Clearly, \(\lvert \beta \rvert > 1\). Hence, the central difference scheme is unconditionally unstable.

But \(\beta\) is close to one if \(\sigma < 1\), meaning the solution slowly grows in amplitude for \(\sigma < 1\)
\section{Implementation}
\label{sec:orgf83e68a}
\begin{verbatim}
#include <iostream>
#include <vector>
\end{verbatim}

\subsection{Forward Difference}
\label{sec:org1479429}
\begin{verbatim}
// This method returns the current state of the solution after n time steps
vector<double> nsol(const int n, const double sigma, const vector<double> &u)
{
    int size = u.size();
    vector<double> u_prev(size,0.0);
    u_prev = u;
    vector<double> u_next(size,0.0);
    for(unsigned int k=0;k<n;k++)
    {
        u_next[size-1] = (1+sigma)*u_prev[size-1] - sigma*u_prev[1];
        for(unsigned int i=0;i<size-1;i++)
        {
            u_next[i] = (1+sigma)*u_prev[i] - sigma*u_prev[i+1];
        }
        u_prev = u_next;
    }
    return u_next;
}
\end{verbatim}
\subsection{Backward Difference}
\label{sec:org9b5da33}
\begin{verbatim}
// This method returns the current state of the solution after n time steps
vector<double> nsol(const int n, const double sigma, const vector<double> &u)
{
    int size = u.size();
    vector<double> u_prev(size,0.0);
    u_prev = u;
    vector<double> u_next(size,0.0);
    for(unsigned int k=0;k<n;k++)
    {
        u_next[0] = (1-sigma)*u_prev[0] + sigma*u_prev[size-2];
        for(unsigned int i=1;i<size;i++)
        {
            u_next[i] = (1-sigma)*u_prev[i] + sigma*u_prev[i-1];
        }
        u_prev = u_next;
    }
    return u_next;
}
\end{verbatim}
\subsection{Central Difference}
\label{sec:org3d7171f}
\begin{verbatim}
// This method returns the current state of the solution after n time steps
vector<double> nsol(const int n, const double sigma, const vector<double> &u)
{
    int size = u.size();
    vector<double> u_prev(size,0.0);
    u_prev = u;
    vector<double> u_next(size,0.0);
    for(unsigned int k=0;k<n;k++)
    {
        u_next[size-1] = u_prev[size-1] - (sigma/2)*u_prev[1] + (sigma/2)*u_prev[size-2];
        u_next[0] = u_prev[0] - (sigma/2)*u_prev[1] + (sigma/2)*u_prev[size-2];
        for(unsigned int i=1;i<size-1;i++)
        {
            u_next[i] = u_prev[i] - (sigma/2)*u_prev[i+1] + (sigma/2)*u_prev[i-1];
        }
        u_prev = u_next;
    }
    return u_next;
}
\end{verbatim}
\end{document}
